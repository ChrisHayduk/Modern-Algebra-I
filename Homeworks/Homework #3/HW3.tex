\documentclass[11pt, reqno]{amsart}
\usepackage[margin=1in]{geometry}    
\geometry{letterpaper}       
%\geometry{landscape}                % Activate for for rotated page geometry
\usepackage[parfill]{parskip}    % Deactivate to begin paragraphs with an indent rather than an empty line
\usepackage{amsfonts, amscd, amssymb, amsthm, amsmath}
\usepackage{pdfsync}  %leaves makers for tex searching
\usepackage{enumerate}
\usepackage{multicol}
\usepackage[pdftex,bookmarks]{hyperref}

\setlength\parindent{0pt}

%%% Theorems %%%--------------------------------------------------------- 
\theoremstyle{plain}
	\newtheorem{thm}{Theorem}[section]
	\newtheorem{lemma}[thm]{Lemma}
	\newtheorem{prop}[thm]{Proposition}
	\newtheorem{cor}[thm]{Corollary}
\theoremstyle{definition}
	\newtheorem*{defn}{Definition}
	\newtheorem{remark}{Remark}
\theoremstyle{example}
	\newtheorem*{example}{Example}


%%% Environments %%%--------------------------------------------------------- 
\newenvironment{ans}{\color{black}\medskip \paragraph*{\emph{Answer}.}}{\hfill \break  $~\!\!$ \dotfill \medskip }
\newenvironment{sketch}{\medskip \paragraph*{\emph{Proof sketch}.}}{ \medskip }
\newenvironment{summary}{\medskip \paragraph*{\emph{Summary}.}}{  \hfill \break  \rule{1.5cm}{0.4pt} \medskip }
\newcommand\Ans[1]{\color{black}\hfill \emph{Answer:} {#1}}


%%% Pictures %%%--------------------------------------------------------- 
%%% If you need to draw pictures, tikzpicture is one good option. Here are some basic things I always use:
\usepackage{tikz}
\usetikzlibrary{arrows}
\tikzstyle{V}=[draw, fill =black, circle, inner sep=0pt, minimum size=2pt]
\newcommand\TikZ[1]{\begin{matrix}\begin{tikzpicture}#1\end{tikzpicture}\end{matrix}}



%%% Color  %%%---------------------------------------------------------
\usepackage{color}
\newcommand{\blue}[1]{{\color{blue}#1}}
\newcommand{\NOTE}[1]{{\color{blue}#1}}
\newcommand{\MOVED}[1]{{\color{gray}#1}}


%%% Alphabets %%%---------------------------------------------------------
%%% Some shortcuts for my commonly used special alphabets and characters.
\def\cA{\mathcal{A}}\def\cB{\mathcal{B}}\def\cC{\mathcal{C}}\def\cD{\mathcal{D}}\def\cE{\mathcal{E}}\def\cF{\mathcal{F}}\def\cG{\mathcal{G}}\def\cH{\mathcal{H}}\def\cI{\mathcal{I}}\def\cJ{\mathcal{J}}\def\cK{\mathcal{K}}\def\cL{\mathcal{L}}\def\cM{\mathcal{M}}\def\cN{\mathcal{N}}\def\cO{\mathcal{O}}\def\cP{\mathcal{P}}\def\cQ{\mathcal{Q}}\def\cR{\mathcal{R}}\def\cS{\mathcal{S}}\def\cT{\mathcal{T}}\def\cU{\mathcal{U}}\def\cV{\mathcal{V}}\def\cW{\mathcal{W}}\def\cX{\mathcal{X}}\def\cY{\mathcal{Y}}\def\cZ{\mathcal{Z}}

\def\AA{\mathbb{A}} \def\BB{\mathbb{B}} \def\CC{\mathbb{C}} \def\DD{\mathbb{D}} \def\EE{\mathbb{E}} \def\FF{\mathbb{F}} \def\GG{\mathbb{G}} \def\HH{\mathbb{H}} \def\II{\mathbb{I}} \def\JJ{\mathbb{J}} \def\KK{\mathbb{K}} \def\LL{\mathbb{L}} \def\MM{\mathbb{M}} \def\NN{\mathbb{N}} \def\OO{\mathbb{O}} \def\PP{\mathbb{P}} \def\QQ{\mathbb{Q}} \def\RR{\mathbb{R}} \def\SS{\mathbb{S}} \def\TT{\mathbb{T}} \def\UU{\mathbb{U}} \def\VV{\mathbb{V}} \def\WW{\mathbb{W}} \def\XX{\mathbb{X}} \def\YY{\mathbb{Y}} \def\ZZ{\mathbb{Z}}  

\def\fa{\mathfrak{a}} \def\fb{\mathfrak{b}} \def\fc{\mathfrak{c}} \def\fd{\mathfrak{d}} \def\fe{\mathfrak{e}} \def\ff{\mathfrak{f}} \def\fg{\mathfrak{g}} \def\fh{\mathfrak{h}} \def\fj{\mathfrak{j}} \def\fk{\mathfrak{k}} \def\fl{\mathfrak{l}} \def\fm{\mathfrak{m}} \def\fn{\mathfrak{n}} \def\fo{\mathfrak{o}} \def\fp{\mathfrak{p}} \def\fq{\mathfrak{q}} \def\fr{\mathfrak{r}} \def\fs{\mathfrak{s}} \def\ft{\mathfrak{t}} \def\fu{\mathfrak{u}} \def\fv{\mathfrak{v}} \def\fw{\mathfrak{w}} \def\fx{\mathfrak{x}} \def\fy{\mathfrak{y}} \def\fz{\mathfrak{z}}
\def\fgl{\mathfrak{gl}}  \def\fsl{\mathfrak{sl}}  \def\fso{\mathfrak{so}}  \def\fsp{\mathfrak{sp}}  
\def\GL{\mathrm{GL}} \def\SL{\mathrm{SL}}  \def\SP{\mathrm{SL}}

\def\<{\langle} \def\>{\rangle}
\usepackage{mathabx}
\def\acts{\lefttorightarrow}
\def\ad{\mathrm{ad}} 
\def\Aut{\mathrm{Aut}}
\def\Ann{\mathrm{Ann}}
\def\dim{\mathrm{dim}} 
\def\End{\mathrm{End}} 
\def\ev{\mathrm{ev}} 
\def\Fr{\mathcal{F}\mathrm{r}}
\def\half{\hbox{$\frac12$}}
\def\Hom{\mathrm{Hom}} 
\def\id{\mathrm{id}} 
\def\sgn{\mathrm{sgn}}  
\def\supp{\mathrm{supp}}  
\def\Tor{\mathrm{Tor}}
\def\tr{\mathrm{tr}} 
\def\vep{\varepsilon}
\def\f{\varphi}


\def\Obj{\mathrm{Obj}}
\def\normeq{\unlhd}
\def\Set{{\cS\mathrm{et}}}
\def\Fin{{\cF\mathrm{inSet}}}
\def\Set{{\cS\mathrm{et}}}
\def\Grp{{\cG\mathrm{rp}}}
\def\Ab{{\cA\mathrm{b}}}
\def\Mod{{\cM\mathrm{od}}}
\def\ab{\mathrm{ab}}
\def\lcm{\mathrm{lcm}}
\def\ZZn{\ZZ/n\ZZ}


%%%%%%%%%%%%%%%%%%%%%%%%%%%%%% 
%%%%%%%%%%%%%%%%%%%%%%%%%%%%%%

\def\HW{3}
\def\DUE{9/25/2020}

\title[Homework \HW]{Homework \HW \\
Math A4900/44900\\
\small Due: \DUE}
\author{}

\begin{document}
\maketitle %%% COMMENT THIS LINE OUT (add a % to the beginning of the line) and UNCOMMENT the following (delete the % symbols) to give yourself a good assignment header:
%\begin{flushright}
%YOUR NAME HERE\\
%Math B4900\\
%Homework \HW\\
%\DUE
%\end{flushright}





\begin{enumerate}[1.]
\item {\bf Group actions}

\begin{enumerate}[(a)] 
\item For some fixed $g \in G$, prove that conjugation by $g$ (i.e.\ the map $G \to G$ defined by $a \mapsto gag^{-1}$) is an automorphism of $G$. Deduce that $a$ and $gag^{-1}$ have the same order (by last week's work), and for any non-empty $S \subseteq G$, the map 
$$S \to gSg^{-1} \quad \text{defined by} \quad s \mapsto gsg^{-1}$$
is also a bijection, so that $|gSg^{-1}| = |S|$. \\
{\small[Recall, even if $A$ and/or $B$ is infinite, we say $|A| = |B|$ exactly when there is a bijection  $A \leftrightarrow B$]}
\item \label{subsets} 
Let $A$ be a non-empty set and let $0<k \leq |A|$. Check that the action of the  symmetric group $S_A$  on the set of size $k$ subsets of $A$ by 
$$\sigma\cdot \{a_1, \dots, a_k\} = \{\sigma(a_1), \dots, \sigma(a_k)\}$$
satisfies the axioms of group actions. \hfill {\small[Similar to the action of $D_{2n}$ on sets from lecture.]}

\item 
 Let $G$ act on a set $A$. Prove that the relation $\sim$ on $A$ defined by 
$$a \sim b \quad  \text{ if and only if } \quad  a = g \cdot b \text{ for some } g \in G$$
is an equivalence relation. 

\smallskip

Note: the equivalence classes with respect to this relation are called \emph{\bf orbits}.

\item Describe the orbits of the action of $S_4$ on $2$-element subsets of $\{1,2,3,4\}$ (as in problem \ref{subsets}).

\end{enumerate}

\item {\bf Cyclic groups}
\begin{enumerate}
\item If $x$ is an element of a finite group $G$ and $|x| = |G|$, prove that $G = \<x\>$. Give an explicit example to show $|x| = |G|$ does not imply $G = \<x\>$ if $G$ is an infinite group.
\item Write $Z_{63} = \<x\>$. For which integers $a$ does the map $\psi_a$ defined by 
$$\psi_a : \bar{1} \to x^a$$
extend to a \emph{well defined homomorphism} from $\ZZ/147 \ZZ$ to $Z_{63}$? Can $\psi_a$ ever be a surjective homomorphism? 
{\small[Take care to remember that the binary operation on the left is $+$ and the binary operation on the right is $\times$: if the image of  $\bar{1}$ is $x^a$, then the image of $\bar{1} + \bar{1} + \cdots + \bar{1} = \ell\bar{1}$ is $(x^{a})^{\ell}$.]}
\item For $a \in \ZZ$, define 
$$\sigma_a: Z_n \to Z_n \quad \text{ by } \quad \sigma_a(x) = x^a \text{ for all } x \in Z_n.$$
Show that $\sigma_a$ is an automorphism of $Z_n$ if and only if $(a, n) = 1$. 
\item Under what circumstances does there exist a non-trivial homomorphism $\f: Z_n \to G$? \\{\small[Note: $\f$ need not be injective or surjective; just well-defined, and not the map $g \mapsto 1$ for all $g$.]}
\item For which $n \in \ZZ_{\ge 1}$ is $(\ZZ/2^n\ZZ)^\times$ cyclic? \hfill {\small[Hint: Try to find more than one subgroup of order 2. Why would this prove $(\ZZ/2^n\ZZ)^\times$ is \emph{not} cyclic? Start by doing some examples.]}
\item Prove that $\QQ \times \QQ$ is not cyclic.
\end{enumerate}


\end{enumerate}
\end{document}