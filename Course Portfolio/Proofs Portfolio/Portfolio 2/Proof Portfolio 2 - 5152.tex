\documentclass[11pt, reqno]{amsart}
\usepackage[margin=1in]{geometry}    
\geometry{letterpaper}       
%\geometry{landscape}                % Activate for for rotated page geometry
\usepackage[parfill]{parskip}    % Deactivate to begin paragraphs with an indent rather than an empty line
\usepackage{amsfonts, amscd, amssymb, amsthm, amsmath}
\usepackage{pdfsync}  %leaves makers for tex searching
\usepackage{enumerate}
\usepackage{multicol}
\usepackage[pdftex,bookmarks]{hyperref}




%%% Theorems %%%--------------------------------------------------------- 
\theoremstyle{plain}
	\newtheorem{thm}{Theorem}[section]
	\newtheorem{lemma}[thm]{Lemma}
	\newtheorem{prop}[thm]{Proposition}
	\newtheorem{cor}[thm]{Corollary}
\theoremstyle{definition}
	\newtheorem*{defn}{Definition}
	\newtheorem{remark}{Remark}
\theoremstyle{example}
	\newtheorem*{example}{Example}


%%% Environments %%%--------------------------------------------------------- 
\newenvironment{ans}{\color{black}\medskip \paragraph*{\emph{Answer}.}}{\hfill \break  $~\!\!$ \dotfill \medskip }
\newenvironment{sketch}{\medskip \paragraph*{\emph{Proof sketch}.}}{ \medskip }
\newenvironment{summary}{\medskip \paragraph*{\emph{Summary}.}}{  \hfill \break  \rule{1.5cm}{0.4pt} \medskip }
\newcommand\Ans[1]{\color{black}\hfill \emph{Answer:} {#1}}


%%% Pictures %%%--------------------------------------------------------- 
%%% If you need to draw pictures, tikzpicture is one good option. Here are some basic things I always use:
\usepackage{tikz}
\usetikzlibrary{arrows}
\tikzstyle{V}=[draw, fill =black, circle, inner sep=0pt, minimum size=2pt]
\newcommand\TikZ[1]{\begin{matrix}\begin{tikzpicture}#1\end{tikzpicture}\end{matrix}}



%%% Color  %%%---------------------------------------------------------
\usepackage{color}
\newcommand{\blue}[1]{{\color{blue}#1}}
\newcommand{\NOTE}[1]{{\color{blue}#1}}
\newcommand{\MOVED}[1]{{\color{gray}#1}}


%%% Alphabets %%%---------------------------------------------------------
%%% Some shortcuts for my commonly used special alphabets and characters.
\def\cA{\mathcal{A}}\def\cB{\mathcal{B}}\def\cC{\mathcal{C}}\def\cD{\mathcal{D}}\def\cE{\mathcal{E}}\def\cF{\mathcal{F}}\def\cG{\mathcal{G}}\def\cH{\mathcal{H}}\def\cI{\mathcal{I}}\def\cJ{\mathcal{J}}\def\cK{\mathcal{K}}\def\cL{\mathcal{L}}\def\cM{\mathcal{M}}\def\cN{\mathcal{N}}\def\cO{\mathcal{O}}\def\cP{\mathcal{P}}\def\cQ{\mathcal{Q}}\def\cR{\mathcal{R}}\def\cS{\mathcal{S}}\def\cT{\mathcal{T}}\def\cU{\mathcal{U}}\def\cV{\mathcal{V}}\def\cW{\mathcal{W}}\def\cX{\mathcal{X}}\def\cY{\mathcal{Y}}\def\cZ{\mathcal{Z}}

\def\AA{\mathbb{A}} \def\BB{\mathbb{B}} \def\CC{\mathbb{C}} \def\DD{\mathbb{D}} \def\EE{\mathbb{E}} \def\FF{\mathbb{F}} \def\GG{\mathbb{G}} \def\HH{\mathbb{H}} \def\II{\mathbb{I}} \def\JJ{\mathbb{J}} \def\KK{\mathbb{K}} \def\LL{\mathbb{L}} \def\MM{\mathbb{M}} \def\NN{\mathbb{N}} \def\OO{\mathbb{O}} \def\PP{\mathbb{P}} \def\QQ{\mathbb{Q}} \def\RR{\mathbb{R}} \def\SS{\mathbb{S}} \def\TT{\mathbb{T}} \def\UU{\mathbb{U}} \def\VV{\mathbb{V}} \def\WW{\mathbb{W}} \def\XX{\mathbb{X}} \def\YY{\mathbb{Y}} \def\ZZ{\mathbb{Z}}  

\def\fa{\mathfrak{a}} \def\fb{\mathfrak{b}} \def\fc{\mathfrak{c}} \def\fd{\mathfrak{d}} \def\fe{\mathfrak{e}} \def\ff{\mathfrak{f}} \def\fg{\mathfrak{g}} \def\fh{\mathfrak{h}} \def\fj{\mathfrak{j}} \def\fk{\mathfrak{k}} \def\fl{\mathfrak{l}} \def\fm{\mathfrak{m}} \def\fn{\mathfrak{n}} \def\fo{\mathfrak{o}} \def\fp{\mathfrak{p}} \def\fq{\mathfrak{q}} \def\fr{\mathfrak{r}} \def\fs{\mathfrak{s}} \def\ft{\mathfrak{t}} \def\fu{\mathfrak{u}} \def\fv{\mathfrak{v}} \def\fw{\mathfrak{w}} \def\fx{\mathfrak{x}} \def\fy{\mathfrak{y}} \def\fz{\mathfrak{z}}
\def\fgl{\mathfrak{gl}}  \def\fsl{\mathfrak{sl}}  \def\fso{\mathfrak{so}}  \def\fsp{\mathfrak{sp}}  
\def\GL{\mathrm{GL}} \def\SL{\mathrm{SL}}  \def\SP{\mathrm{SL}}

\def\<{\langle} \def\>{\rangle}
\usepackage{mathabx}
\def\acts{\lefttorightarrow}
\def\ad{\mathrm{ad}} 
\def\Aut{\mathrm{Aut}}
\def\Ann{\mathrm{Ann}}
\def\dim{\mathrm{dim}} 
\def\End{\mathrm{End}} 
\def\ev{\mathrm{ev}} 
\def\Fr{\mathcal{F}\mathrm{r}}
\def\half{\hbox{$\frac12$}}
\def\Hom{\mathrm{Hom}} 
\def\id{\mathrm{id}} 
\def\sgn{\mathrm{sgn}}  
\def\supp{\mathrm{supp}}  
\def\Tor{\mathrm{Tor}}
\def\tr{\mathrm{tr}} 
\def\vep{\varepsilon}
\def\f{\varphi}


\def\Obj{\mathrm{Obj}}
\def\normeq{\unlhd}
\def\Set{{\cS\mathrm{et}}}
\def\Fin{{\cF\mathrm{inSet}}}
\def\Set{{\cS\mathrm{et}}}
\def\Grp{{\cG\mathrm{rp}}}
\def\Ab{{\cA\mathrm{b}}}
\def\Mod{{\cM\mathrm{od}}}
\def\ab{\mathrm{ab}}
\def\lcm{\mathrm{lcm}}
\def\ZZn{\ZZ/n\ZZ}


\newcommand{\ProblemID}[2]{{\def\arraystretch{1.5}
	\begin{tabular}{|lr|}\hline
	Problem: & \bf #1\\\hline
	No.\ stars:& \bf #2\\\hline\end{tabular}}}


\newcommand{\Rubric}[1]{$~$\\\vfill \hfill{\def\arraystretch{1.75}\begin{tabular} {|c|c|} \hline
#1 & Points Possible  \\ \hline \hline
complete & \hspace{3mm} 0 \hspace{3mm} 1 \hspace{3mm} 2 \hspace{3mm} 
			3 \hspace{3mm} 4 \hspace{3mm} 5 \hspace{3mm} \\ \hline
mathematically valid & \hspace{3mm} 0 \hspace{3mm} 1 \hspace{3mm} 2 \hspace{3mm} 
			3 \hspace{3mm} 4 \hspace{3mm} 5 \hspace{3mm} \\ \hline
readable/fluent & \hspace{3mm} 0 \hspace{3mm} 1 \hspace{3mm} 2 \hspace{3mm} 
			3 \hspace{3mm} 4 \hspace{3mm} 5 \hspace{3mm} \\ \hline
Total:& \qquad\qquad\qquad(out of 15)\\
\hline
\end{tabular}}
\pagebreak}

\title{Proof portfolio draft, round 2 -- \IDNUMBER}
\author{}
\usepackage{fancyhdr}
\pagestyle{fancy}
\fancyhf{}
\rhead{\IDNUMBER}
\lhead{Proof portfolio draft, round 2}
\rfoot{\thepage}


%%%%%%%%%%%%%%%%%%%%%%%%%%%%%% 
%%%%%%%%%%%%%%%%%%%%%%%%%%%%%%


\def\IDNUMBER{5152}%replace "YOUR-ID-NUMBER" with the ID number given to you by Prof Daugherty (NOT your CCNY emplid, or any other number).

\begin{document}
\begin{flushright}
\fbox{ID: {\bf \IDNUMBER}}\\\smallskip
Math A4900\\
Proof portfolio draft, Round 2\\
November 15, 2020
\end{flushright}
\medskip
\hrule
\medskip

%%%%%%%%%%%%%%%%%%%%%%%%%%%%%%%%
%%%%%%%%% Copy and past one of these %%%%%%%%%
%%%%%%%%% for each problem you rewrite %%%%%%%%
%%%%%%%%%%%%%%%%%%%%%%%%%%%%%%%%


\hbox{\begin{minipage}{5in}
\noindent {\bf Statement:} 
Prove that every finitely generated subgroup of $\QQ$ is cyclic.\end{minipage} \hspace{.3in} {\begin{minipage}{1.1in}
\ProblemID
		{4A}%PUT PROBLEM NUMBER HERE, e.g. 4A in place of 0X.
		{1}%PUT NUMBER OF STARS HERE, e.g. 2 in place of 0.
\end{minipage}}}

\begin{proof}
Let $H$ be a finitely generated subgroup of $\QQ$ and suppose that there is a finite set $\QQ$ such that $H = \langle A \rangle$. Now consider $k$, the product of all the denominators that appear in $A$. Then every element $a/b \in A$ can be re-written as $\frac{a \cdot k/b}{b \cdot k/b} = \frac{a \cdot k/b}{k}$ since $b$ is in the product that yields $k$ and hence is a divisor of $k$. Thus, we can rewrite every fraction in $A$ as a fraction with denominator $k$. That is, every fraction in $A$ can be written as $n/k$ for some $n \in \ZZ$. This lets us conclude that, $$H = \langle A \rangle \leq \langle 1/k \rangle$$

Thus, by Theorem 7 in $\S 2.3$ of DF, we have that $H$ is cyclic since $\langle 1/k \rangle$ is cyclic.
\end{proof}

\Rubric{}
\newpage
%%%%%%%%%%%%%%%%%%%%%%%%%%%%%%%%
%%%%%%%%%%%%%%%%%%%%%%%%%%%%%%%%

\hbox{\begin{minipage}{5in}
\noindent {\bf Statement:} 
Prove that if $G/Z(G)$ is cyclic, then $G$ is abelian.
\end{minipage} \hspace{.3in} {\begin{minipage}{1.1in}
\ProblemID
		{5A}%PUT PROBLEM NUMBER HERE, e.g. 4A in place of 0X.
		{2}%PUT NUMBER OF STARS HERE, e.g. 2 in place of 0.
\end{minipage}}}

\begin{proof}
Suppose $G/Z(G)$ is cyclic. That is, there is an element $x \in G$ such that $G/Z(G) = \langle xZ(G) \rangle$. That is, for every $g \in G$, we can rewrite $gZ(G)$ as $x^{\ell}Z(G)$ for some $\ell \in \ZZ$. Hence, $gZ(G) = x^{\ell}Z(G)$. By Proposition 4, we then have,
\begin{align*}
x^{-\ell}g \in Z(G)
\end{align*}

But if $x^{-\ell}g \in Z(G)$, then,
\begin{align*}
x^{\ell}(x^{-\ell}g) = g \in x^{\ell}Z(G)
\end{align*}

Hence, $g = x^{\ell}z$ for some $z \in Z(G)$ (in particular, $z = x^{\ell}g$). Since $g$ and $\ell$ were arbitrary, this holds for every element $g \in G$. Now let us fix $g_1, g_2 \in G$ such that $g_1 = x^az_1$ and $g_2 = x^bz_2$ where $z_1, z_2 \in Z(G)$. Since $Z(G)$ is the set of elements that commute with everything in $G$, we have,
\begin{align*}
g_1g_2 &= x^az_1 \cdot x^bz_2\\
&= x^ax^bz_1z_2\\
&= x^{a+b}z_2z_1\\
&= x^{b+a}z_2z_1\\
&= x^bx^az_2z_1\\
&= x^bz_2x^az_1\\
&= g_2g_1
\end{align*}

Thus, $G$ is an abelian group.
\end{proof}

\Rubric{}
\newpage
%%%%%%%%%%%%%%%%%%%%%%%%%%%%%%%%
%%%%%%%%%%%%%%%%%%%%%%%%%%%%%%%%

\hbox{\begin{minipage}{5in}
\noindent {\bf Statement:} 
Prove that if $H$ and $K$ are finite subgroups of $G$ whose orders are relatively prime, then $H \cap K = 1$.
\end{minipage} \hspace{.3in} {\begin{minipage}{1.1in}
\ProblemID
		{5B}%PUT PROBLEM NUMBER HERE, e.g. 4A in place of 0X.
		{1}%PUT NUMBER OF STARS HERE, e.g. 2 in place of 0.
\end{minipage}}}

\begin{proof}
Suppose $H$ and $K$ are finite subgroups of $G$ where $|H| = p$, $|K| = q$ with $q$ and $p$ relatively prime. By Proposition 13 on page 93 of Dummit \& Foote, we have that,
\begin{align*}
|HK| &= \frac{|H||K|}{|H \cap K|}\\
&= \frac{pq}{|H \cap K|}
\end{align*}

Suppose without loss of generality that $p \geq q$. Then we have that $|H \cap K| \leq |K| = q$. Now note that $\frac{pq}{|H \cap K|}$ must yield an integer answer. However, we have that there are no common factors of $p$ and $q$ in the set $\{2, 3, 4, \cdots, q-1\}$. Thus, our choices for $|H \cap K|$ are $1$ and $q$. We know that $|H \cap K| = q$ if $K \le H$. However, if $K \le H$, then by Lagrange's Theorem, $|K| = q$ divides $|H| = p$. Since $p, q$ are relatively prime, this is not possible. Hence, $|H \cap K| = 1$.

\par
Now, since both $H$ and $K$ are subgroups of $G$, we know that they must both contain the identity element $1$. Hence, $1 \in H \cap K$. Since $|H \cap K| = 1$, we have that the identity must be the only element of $H \cap K$. Thus, $H \cap K = 1$.
\end{proof}

\Rubric{}
\newpage
%%%%%%%%%%%%%%%%%%%%%%%%%%%%%%%%
%%%%%%%%%%%%%%%%%%%%%%%%%%%%%%%%

\hbox{\begin{minipage}{5in}
\noindent {\bf Statement:} 
Let $x, y$ be distinct $3$-cycles in $S_5$. Show that $\langle x, y \rangle$ is isomorphic to one of $Z_3$, $A_4$, or $A_5$.
\end{minipage} \hspace{.3in} {\begin{minipage}{1.1in}
\ProblemID
		{6B.I}%PUT PROBLEM NUMBER HERE, e.g. 4A in place of 0X.
		{2}%PUT NUMBER OF STARS HERE, e.g. 2 in place of 0.
\end{minipage}}}

\begin{proof}
Let $x, y$ be distinct $3$-cycles in $S_5$. There are $3$ possible cases: $y = x^{-1}$, $y \neq x^{-1}$ and they overlap at one element, in which case they permute all $5$ elements, or $y \neq x^{-1}$ and they overlap at two elements, in which case they permute $4$ elements and fix the fifth element.

\par
Let us examine the first case, $y = x^{-1}$. Since $y = x^{-1}$, we must have that $y^{-1} \in \langle x \rangle$. Hence, $\langle x, y \rangle = \langle x \rangle$. Note that if we let $x = (a \ b \ c)$, we have that,
\begin{align*}
x^2 &= (a \ c \ b)\\
x^3 &= 1
\end{align*}

So $\langle x \rangle$ is a cyclic group of order $3$. $Z_3$ is also a cyclic group of order $3$ and, by Theorem 4 in $\S 2.3$ of Dummit and Foote, we have that any two cyclic groups of the same order are isomorphic. Hence, in this case, $\langle x, y \rangle = \langle x \rangle \cong Z_3$.

\par
Now assume $y \neq x^{-1}$ and that $x, y$ overlap at one element. Hence, $x$ and $y$ permute all $5$ elements in $\{1, 2, 3, 4, 5\}$.

\end{proof}

\Rubric{}
\newpage
%%%%%%%%%%%%%%%%%%%%%%%%%%%%%%%%
%%%%%%%%%%%%%%%%%%%%%%%%%%%%%%%%

\hbox{\begin{minipage}{5in}
\noindent {\bf Statement:} 
Show that if $H$ has finite index $n$ in $G$, then there is a normal subgroup $K \normeq G$ with $K \leq H$ and $|G : K| \leq n!$.
\end{minipage} \hspace{.3in} {\begin{minipage}{1.1in}
\ProblemID
		{7A}%PUT PROBLEM NUMBER HERE, e.g. 4A in place of 0X.
		{2}%PUT NUMBER OF STARS HERE, e.g. 2 in place of 0.
\end{minipage}}}

\begin{proof}
Suppose $H$ has finite index $n$ in $G$. That is, $|G : H| = n$ or, equivalently, $|G/H| = n$. Note that $G/H$ is the set of left cosets of $H$ in $G$. Now, since $|G : H| = n$, we can label the distinct left cosets of $H$ in $G$ by $a_1H, a_2H, \ldots, a_nH$. For each $g \in G$, we can then think of the action of left multiplication as a permutation $\sigma_g$ of the indices $1, \ldots, n$. That is, $\sigma_g \in S_n$. Then if we define the homomorphism $\varphi: g \mapsto \sigma_g$, we have that $\varphi: G \to S_{n}$. Now the kernel of $\varphi$ is given by $K = \{g \in G \ | \ gaH = aH \text{ for all } a \in G\}$.

\par
Suppose $g \in K$. Then we have $g1H = gH = H$, and hence $g \in H$. This gives us that $K \subset H$. Moreover, by the First Isomorphism Theorem, we now have that $K \normeq G$ and $G/K \cong \varphi(G)$. Since $\varphi: G \to S_n$, we have that $\varphi(G) \leq S_n$. In addition, $|S_n| = n!$, so we have that $\varphi(G) \leq |S_n| = n!$. Lastly, by Corollary 17 in $\S 3.3$ of Dummit and Foote, we have $|G/K| = |\varphi(G)| \leq n!$, as required.
\end{proof}

\Rubric{}
\end{document}