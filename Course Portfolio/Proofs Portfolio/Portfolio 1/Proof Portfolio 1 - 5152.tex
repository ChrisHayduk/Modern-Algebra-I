\documentclass[11pt, reqno]{amsart}
\usepackage[margin=1in]{geometry}    
\geometry{letterpaper}       
%\geometry{landscape}                % Activate for for rotated page geometry
\usepackage[parfill]{parskip}    % Deactivate to begin paragraphs with an indent rather than an empty line
\usepackage{amsfonts, amscd, amssymb, amsthm, amsmath}
\usepackage{pdfsync}  %leaves makers for tex searching
\usepackage{enumerate}
\usepackage{multicol}
\usepackage[pdftex,bookmarks]{hyperref}


\usepackage[pagewise, displaymath, mathlines]{lineno}
\linenumbers


%%% Theorems %%%--------------------------------------------------------- 
\theoremstyle{plain}
	\newtheorem{thm}{Theorem}[section]
	\newtheorem{lemma}[thm]{Lemma}
	\newtheorem{prop}[thm]{Proposition}
	\newtheorem{cor}[thm]{Corollary}
\theoremstyle{definition}
	\newtheorem*{defn}{Definition}
	\newtheorem{remark}{Remark}
\theoremstyle{example}
	\newtheorem*{example}{Example}


%%% Environments %%%--------------------------------------------------------- 
\newenvironment{ans}{\color{black}\medskip \paragraph*{\emph{Answer}.}}{\hfill \break  $~\!\!$ \dotfill \medskip }
\newenvironment{sketch}{\medskip \paragraph*{\emph{Proof sketch}.}}{ \medskip }
\newenvironment{summary}{\medskip \paragraph*{\emph{Summary}.}}{  \hfill \break  \rule{1.5cm}{0.4pt} \medskip }
\newcommand\Ans[1]{\color{black}\hfill \emph{Answer:} {#1}}


%%% Pictures %%%--------------------------------------------------------- 
%%% If you need to draw pictures, tikzpicture is one good option. Here are some basic things I always use:
\usepackage{tikz}
\usetikzlibrary{arrows}
\tikzstyle{V}=[draw, fill =black, circle, inner sep=0pt, minimum size=2pt]
\newcommand\TikZ[1]{\begin{matrix}\begin{tikzpicture}#1\end{tikzpicture}\end{matrix}}



%%% Color  %%%---------------------------------------------------------
\usepackage{color}
\newcommand{\blue}[1]{{\color{blue}#1}}
\newcommand{\NOTE}[1]{{\color{blue}#1}}
\newcommand{\MOVED}[1]{{\color{gray}#1}}


%%% Alphabets %%%---------------------------------------------------------
%%% Some shortcuts for my commonly used special alphabets and characters.
\def\cA{\mathcal{A}}\def\cB{\mathcal{B}}\def\cC{\mathcal{C}}\def\cD{\mathcal{D}}\def\cE{\mathcal{E}}\def\cF{\mathcal{F}}\def\cG{\mathcal{G}}\def\cH{\mathcal{H}}\def\cI{\mathcal{I}}\def\cJ{\mathcal{J}}\def\cK{\mathcal{K}}\def\cL{\mathcal{L}}\def\cM{\mathcal{M}}\def\cN{\mathcal{N}}\def\cO{\mathcal{O}}\def\cP{\mathcal{P}}\def\cQ{\mathcal{Q}}\def\cR{\mathcal{R}}\def\cS{\mathcal{S}}\def\cT{\mathcal{T}}\def\cU{\mathcal{U}}\def\cV{\mathcal{V}}\def\cW{\mathcal{W}}\def\cX{\mathcal{X}}\def\cY{\mathcal{Y}}\def\cZ{\mathcal{Z}}

\def\AA{\mathbb{A}} \def\BB{\mathbb{B}} \def\CC{\mathbb{C}} \def\DD{\mathbb{D}} \def\EE{\mathbb{E}} \def\FF{\mathbb{F}} \def\GG{\mathbb{G}} \def\HH{\mathbb{H}} \def\II{\mathbb{I}} \def\JJ{\mathbb{J}} \def\KK{\mathbb{K}} \def\LL{\mathbb{L}} \def\MM{\mathbb{M}} \def\NN{\mathbb{N}} \def\OO{\mathbb{O}} \def\PP{\mathbb{P}} \def\QQ{\mathbb{Q}} \def\RR{\mathbb{R}} \def\SS{\mathbb{S}} \def\TT{\mathbb{T}} \def\UU{\mathbb{U}} \def\VV{\mathbb{V}} \def\WW{\mathbb{W}} \def\XX{\mathbb{X}} \def\YY{\mathbb{Y}} \def\ZZ{\mathbb{Z}}  

\def\fa{\mathfrak{a}} \def\fb{\mathfrak{b}} \def\fc{\mathfrak{c}} \def\fd{\mathfrak{d}} \def\fe{\mathfrak{e}} \def\ff{\mathfrak{f}} \def\fg{\mathfrak{g}} \def\fh{\mathfrak{h}} \def\fj{\mathfrak{j}} \def\fk{\mathfrak{k}} \def\fl{\mathfrak{l}} \def\fm{\mathfrak{m}} \def\fn{\mathfrak{n}} \def\fo{\mathfrak{o}} \def\fp{\mathfrak{p}} \def\fq{\mathfrak{q}} \def\fr{\mathfrak{r}} \def\fs{\mathfrak{s}} \def\ft{\mathfrak{t}} \def\fu{\mathfrak{u}} \def\fv{\mathfrak{v}} \def\fw{\mathfrak{w}} \def\fx{\mathfrak{x}} \def\fy{\mathfrak{y}} \def\fz{\mathfrak{z}}
\def\fgl{\mathfrak{gl}}  \def\fsl{\mathfrak{sl}}  \def\fso{\mathfrak{so}}  \def\fsp{\mathfrak{sp}}  
\def\GL{\mathrm{GL}} \def\SL{\mathrm{SL}}  \def\SP{\mathrm{SL}}

\def\<{\langle} \def\>{\rangle}
\usepackage{mathabx}
\def\acts{\lefttorightarrow}
\def\ad{\mathrm{ad}} 
\def\Aut{\mathrm{Aut}}
\def\Ann{\mathrm{Ann}}
\def\dim{\mathrm{dim}} 
\def\End{\mathrm{End}} 
\def\ev{\mathrm{ev}} 
\def\Fr{\mathcal{F}\mathrm{r}}
\def\half{\hbox{$\frac12$}}
\def\Hom{\mathrm{Hom}} 
\def\id{\mathrm{id}} 
\def\sgn{\mathrm{sgn}}  
\def\supp{\mathrm{supp}}  
\def\Tor{\mathrm{Tor}}
\def\tr{\mathrm{tr}} 
\def\vep{\varepsilon}
\def\f{\varphi}


\def\Obj{\mathrm{Obj}}
\def\normeq{\unlhd}
\def\Set{{\cS\mathrm{et}}}
\def\Fin{{\cF\mathrm{inSet}}}
\def\Set{{\cS\mathrm{et}}}
\def\Grp{{\cG\mathrm{rp}}}
\def\Ab{{\cA\mathrm{b}}}
\def\Mod{{\cM\mathrm{od}}}
\def\ab{\mathrm{ab}}
\def\lcm{\mathrm{lcm}}
\def\ZZn{\ZZ/n\ZZ}


\newcommand{\ProblemID}[2]{{\def\arraystretch{1.5}
	\begin{tabular}{|lr|}\hline
	Problem: & \bf #1\\\hline
	No.\ stars:& \bf #2\\\hline\end{tabular}}}


\newcommand{\Rubric}[1]{$~$\\\vfill \hfill{\def\arraystretch{1.75}\begin{tabular} {|c|c|} \hline
#1 & Points Possible  \\ \hline \hline
complete & \hspace{3mm} 0 \hspace{3mm} 1 \hspace{3mm} 2 \hspace{3mm} 
			3 \hspace{3mm} 4 \hspace{3mm} 5 \hspace{3mm} \\ \hline
mathematically valid & \hspace{3mm} 0 \hspace{3mm} 1 \hspace{3mm} 2 \hspace{3mm} 
			3 \hspace{3mm} 4 \hspace{3mm} 5 \hspace{3mm} \\ \hline
readable/fluent & \hspace{3mm} 0 \hspace{3mm} 1 \hspace{3mm} 2 \hspace{3mm} 
			3 \hspace{3mm} 4 \hspace{3mm} 5 \hspace{3mm} \\ \hline
Total:& \qquad\qquad\qquad(out of 15)\\
\hline
\end{tabular}}
\pagebreak}

\title{Proof portfolio draft, round 1 -- \IDNUMBER}
\author{}
\usepackage{fancyhdr}
\pagestyle{fancy}
\fancyhf{}
\rhead{\IDNUMBER}
\lhead{Proof portfolio draft, round 1}
\rfoot{\thepage}


%%%%%%%%%%%%%%%%%%%%%%%%%%%%%% 
%%%%%%%%%%%%%%%%%%%%%%%%%%%%%%


\def\IDNUMBER{5152}%replace "YOUR-ID-NUMBER" with the ID number given to you by Prof Daugherty (NOT your CCNY emplid, or any other number).

\begin{document}
\begin{flushright}
\fbox{ID: {\bf \IDNUMBER}}\\\smallskip
Math A4900\\
Proof portfolio draft, Round 1\\
October 4, 2020
\end{flushright}
\medskip
\hrule
\medskip

%%%%%%%%%%%%%%%%%%%%%%%%%%%%%%%%
%%%%%%%%% Copy and past one of these %%%%%%%%%
%%%%%%%%% for each problem you rewrite %%%%%%%%
%%%%%%%%%%%%%%%%%%%%%%%%%%%%%%%%


\hbox{\begin{minipage}{5in}
\noindent {\bf Statement:} 
Let $G$ be a group and let $x \in G$. If $|x| = n < \infty$, prove that the elements $1, x, x^2, \cdots, x^{n-1}$ are all distinct. Deduce that $|x| = |\langle x \rangle|$
\end{minipage} \hspace{.3in} {\begin{minipage}{1.1in}
\ProblemID
		{1C}%PUT PROBLEM NUMBER HERE, e.g. 2A in place of 0X.
		{2}%PUT NUMBER OF STARS HERE, e.g. 3 in place of 0.
\end{minipage}}}

\begin{proof}
Let $|x| = n < \infty$. Now suppose that there are numbers $m, k \in \ZZ$ with $0 \leq k < m \leq n-1$ such that $x^m = x^k$.\\

Then we have that,
\begin{align*}
x^k \cdot x &= x^m \cdot x\\
x^k \cdot x^2 &= x^m \cdot x^2\\
x^k \cdot x^3 &= x^m \cdot x^3\\
&\vdots\\
x^k \cdot x^{n-m} &= x^m \cdot x^{n-m}
\end{align*}

However, on the right side of the equality, we have
\begin{align*}
x^m \cdot x^{n-m} &= x^{m+n-m}\\
&= x^n\\
&= e
\end{align*}

This implies that,
\begin{align*}
x^k \cdot x^{n-m} &= x^{k+n-m}\\
&= e
\end{align*}

where $k+n-m \in \ZZ$ and $0 < k+n-m < m+n-m = n$.\\

However, we know that the order of $x$ is $n$, which is defined to be the smallest positive integer of $x$ that yields the identity element. Hence, we have a contradiction and thus $e, x, x^2, \dots, x^{n-1}$ are all distinct.\\

Now consider $\langle x \rangle$. We know that each $x^c$ is distinct for every $c \in \ZZ$ such that $0 \leq k \leq n-1$. Now fix an $m \in \ZZ$ such that $m \geq n$. Choose $k \in \NN$ as the greatest positive integer such that $m \geq kn$. Then we have,
\begin{align*}
x^m &= x^{kn + (m - kn)}\\
&= x^{kn}x^{m - kn}\\
&= x^{m-kn}
\end{align*}

Note that $0 \leq m - kn$ since $m \geq kn$. In addition, $m - kn < n$ because, if $m - kn \geq n$, it would mean that $(k+1)n \leq m$. But we chose $k$ such that it was the greatest positive integer with $m \geq kn$, so this is not possible.\\

Hence we have that $0 \leq m - kn < n$, and so $x^{m - kn} \in \{1, x, x^2, \cdots, x^{n-1}\}$. Since $m \geq n$ was an arbitrary integer, this holds for any $x^m$ with $m \geq n$. Thus, for any $a \in \ZZ$, we have that,
\begin{align*}
x^a \in \{1, x, x^2, \cdots, x^{n-1}\}
\end{align*} 

and so $| \langle x \rangle | = n = |x|$.
\end{proof}

\Rubric{}
%%%%%%%%%%%%%%%%%%%%%%%%%%%%%%%%
%%%%%%%%%%%%%%%%%%%%%%%%%%%%%%%%

\hbox{\begin{minipage}{5in}
\noindent {\bf Statement:} 
Show that if $s_1 =  s$ and $s_2 = sr$, then those together with the relations 
$$s_1^2 = s_2^2 = (s_1s_2)^n = 1$$
forms and alternative presentation of $D_{2n}$ 
\end{minipage} \hspace{.3in} {\begin{minipage}{1.1in}
\ProblemID
		{1D}%PUT PROBLEM NUMBER HERE, e.g. 2A in place of 0X.
		{2}%PUT NUMBER OF STARS HERE, e.g. 3 in place of 0.
\end{minipage}}}

\begin{proof}
By the relations given above we have that,
\begin{align*}
s_1^2 = 1 = s^2
\end{align*}

Moreover, we have that
\begin{align*}
s_2^2 = 1 = srsr
\end{align*}

This implies that $sr = (sr)^{-1} = r^{-1}s^{-1}$.\\

However, we know that $s^{-1} = s$ since $s^2 = 1$, so we have
\begin{align*}
sr = (sr)^{-1} &= r^{-1}s^{-1}\\
&= r^{-1}s
\end{align*}

We are also given that $(s_1s_2)^n = 1$. That is,
\begin{align*}
(s_1s_2)^n &= (ssr)^n\\
&= r^n\\
&= 1
\end{align*}

Hence, the elements $s_1$ and $s_2$ together with the relations shown above fully describe the initial presentation of $D_{2n}$.
\end{proof}

\Rubric{}

%%%%%%%%%%%%%%%%%%%%%%%%%%%%%%%%
%%%%%%%%%%%%%%%%%%%%%%%%%%%%%%%%

\hbox{\begin{minipage}{5in}
\noindent {\bf Statement:} 
Prove that if $H$ and $K$ are subgroups of $G$, then so is $H \cap K$. 
On the other hand, prove $H \cup K$ is a subgroup if and only if $H \subseteq K$ or $K \subseteq H$.
\end{minipage} \hspace{.3in} {\begin{minipage}{1.1in}
\ProblemID
		{2A}%PUT PROBLEM NUMBER HERE, e.g. 2A in place of 0X.
		{2}%PUT NUMBER OF STARS HERE, e.g. 3 in place of 0.
\end{minipage}}}

\begin{proof}
Suppose $H, K \leqslant G$. Consider $H \cap K$. Note that $1 \in H, K$ by the definition of groups, so $1 \in H \cap K$. Hence, $H \cap K \neq \emptyset$. Now let $x, y \in H \cap K$. Then $x, y \in H$ and $x, y \in K$, both of which are groups. Hence, $y^{-1} \in H$ and $y^{-1} \in K$, which implies $xy^{-1} \in H$ and $xy^{-1} \in K$. Thus, $xy^{-1} \in H \cap K$. As a result, $H \cap K$ satisfies the subgroup criterion and is hence a subgroup of $G$.\\

Now consider $H \cup K$. Suppose for contraposition that $H \not\subset K$ and $K \not\subset H$. Then $\exists x \in H$ such that $x \not\in K$ and $\exists y \in K$ such that $y \not\in H$. Then we have $y^{-1} \not\in H$ and $x \not\in K$, so $xy^{-1} \not\in H, K$. Hence $xy^{-1} \not\in H \cup K$ and so $H \cup K$ does not satisfy the subgroup criterion. As a result, we have that if $H \cup K$ is a subgroup of $G$, then $H \subset K$ or $K \subset H$.\\

Now for the other direction of the proof. Suppose $H \subset K$. Then $\forall \; x \in H$ we have $x \in K$. Hence, $H \cup K = K$. Since $K \leqslant G$, we have $H \cup K \leqslant G$ as well.\\

Suppose $K \subset H$. Then $\forall \; x \in K$ we have $x \in H$. Hence, $H \cup K = H$. Since $H \leqslant G$, we have $H \cup K \leqslant G$ as well. Thus, we have proved that if $H \subset K$ or $K \subset H$, then $H \cup K$ is a subgroup of $G$.
\end{proof}

\Rubric{}

%%%%%%%%%%%%%%%%%%%%%%%%%%%%%%%%
%%%%%%%%%%%%%%%%%%%%%%%%%%%%%%%%

\hbox{\begin{minipage}{5in}
\noindent {\bf Statement:} 
Prove that $\RR^\times$ is not isomorphic to $\CC^\times$,
forms and alternative presentation of $D_{2n}$ 
\end{minipage} \hspace{.3in} {\begin{minipage}{1.1in}
\ProblemID
		{2C}%PUT PROBLEM NUMBER HERE, e.g. 2A in place of 0X.
		{1}%PUT NUMBER OF STARS HERE, e.g. 3 in place of 0.
\end{minipage}}}

\begin{proof}
There are only $2$ elements in $\RR^\times$ with order less than $\infty$: $|1| = 1$ and $|-1| = 2$. However, there are $4$ in $\CC^\times$: $|1| = 1$, $|-1| = 2$, $|i| = 4$, $|-i| = 4$.\\

Since there is no element $\RR^\times$ with order $4$, for any potential isomorphism $\varphi: \CC^\times \to \RR^\times$, we have
\begin{align*}
|i| \neq |\varphi(i)|
\end{align*}

Hence, $\RR^\times$ and $\CC^\times$ are not isomorphic.

\end{proof}

\Rubric{}

%%%%%%%%%%%%%%%%%%%%%%%%%%%%%%%%
%%%%%%%%%%%%%%%%%%%%%%%%%%%%%%%%

\hbox{\begin{minipage}{5in}
\noindent {\bf Statement:} 
 Let $G$ act on a set $A$. Prove that the relation $\sim$ on $A$ defined by 
$$a \sim b \quad  \text{ if and only if } \quad  a = g \cdot b \text{ for some } g \in G$$
is an equivalence relation. 
\end{minipage} \hspace{.3in} {\begin{minipage}{1.1in}
\ProblemID
		{3B}%PUT PROBLEM NUMBER HERE, e.g. 2A in place of 0X.
		{1}%PUT NUMBER OF STARS HERE, e.g. 3 in place of 0.
\end{minipage}}}

\begin{proof}
We need to check that this relation is reflexive, symmetric, and transitive. We will start with reflexivity. Since $G$ is a group, then $1 \in G$ and so we have 
\begin{align*}
a = 1 \cdot a
\end{align*}

Hence, we have $a \sim a$. Now let $a, b \in A$ and suppose $a \sim b$. Then,
\begin{align*}
a = g \cdot b
\end{align*}

for some $g \in G$. Since $G$ is a group, we have $g^{-1} \in G$ and hence
\begin{align*}
g^{-1} \cdot a &= g^{-1} \cdot (g \cdot b)
\end{align*}

By properties of group actions, we can write
\begin{align*}
g^{-1} \cdot a &= (g^{-1}g) \cdot b\\
&= b
\end{align*}

So we have that $b \sim a$ since $g^{-1} \in G$. Hence, the relation is symmetric.\\

Now let $a, b, c \in A$. Suppose $a \sim b$ and $b \sim c$. Then we have,
\begin{align*}
a = g _1 \cdot b
\end{align*}

and 
\begin{align*}
b = g_2 \cdot c
\end{align*}

for some $g_1, g_2 \in G$. We can use our equation for $b$ and the properties of group action to rewrite $a$ as
\begin{align*}
a = (g_1g_2) \cdot c
\end{align*}

Since $g_1g_2 \in G$, we have that $a \sim c$ and so the relation is transitive. Hence, this is an equivalence relation.
\end{proof}

\Rubric{}
\end{document}